\chapter{Исследовательский раздел}
\label{ch:research}
%
% % В начале раздела  можно напомнить его цель
%

\section{Почему мобильное приложение?}
\label{sec:whyApp}

Мобильные приложения могут быть инструментом для быстрой доставки информации, чего нельзя добиться при помощи обычного веб-приложения.

Индустрия мобильных устройств очень быстро развивается.
На смену старым устройствам приходят более новые, современные и обладающие большим спектром возможностей.
Количество пользователей с каждым годом растёт.
сейчас смартфоны есть почти у всех студентов и они редко с ними расстаются надолго.
Конечно, важную роль играет программная составляющая "--- мобильные приложения.
Они существуют совершенно разной направленности:
\begin{itemize}
  \item развлекательные (игры, музыкальные и видео проигрыватели и т.д.);
  \item коммуникационные (мессенджеры, навигаторы и т.д.);
  \item справочные (словари, базы знаний);
  \item прикладные (все остальные от графического редактора до калькулятора).
\end{itemize}

Кроме того, популярность мобильных приложений повлекла за собой появление мощных инструментов разработки, большого количества библиотек и фреймфорков, что, в свою очередь, сделало разработку приложений быстрой, лёгкой и продуктивной.


\section{Обзор аналогичных решений}
\label{sec:analogs}

Был произведен поиск существующих приложений в Windows Phone Store, Google Play, Apple App Store, на GitHub, и было найдено только 3 аналога.
К ним можно, так же, добавить сам сайт ОРИОКС.

Рассмотрим преимущества и недостатки каждого решения по отдельности.
В качестве критериев будем брать:
\begin{itemize}
  \item способ получения данных;
  \item возможность push-уведомлений;
  \item возможность просмотра расписания;
  \item возможность просмотра текущих предметов;
  \item возможность просмотра успеваемости;
  \item возможность просмотра списка долгов;
  \item возможность просмотра списка пересдач;
  \item наличие графического интерфейса;
  \item оффлайн доступ.
\end{itemize}
\Define{Push-уведомления}{способов рассылки уведомлений. У конечного пользователя такие уведомления выглядят как небольшое всплывающее окошко на экране телефона или в браузере}

\subsection{Сайт ОРИОКС}
\label{subsec:orioks}

\Abbrev{БД}{база данных}
\Abbrev{HTML}{hypertext markup language "--- язык гипертекстовой разметки}
Сайт ОРИОКС получает информацию напрямую из БД, но конечному пользователю для работы сайта нужно подгрузить таблицы стилей и HTML разметку.


\subsection{Приложение ``Расписание для МИЭТ''}
\label{subsec:appMietSchedule}

\subsection{Приложение ``ОРИОКС Live''}
\label{subsec:appOrioksLive}

\subsection{Телеграм-бот ``Open Orioks''}
\label{subsec:botOpenOrioks}

\section{Обзор популярных мобильных платформ}
\label{sec:platforms}

\section{Исследование структуры ОРИОКС}
\label{sec:orioksStructure}

\section{Входные и выходные данные}
\label{sec:io}

\section{Постановка задачи}
\label{sec:problem}
