\chapter{Исследовательский раздел}
\label{ch:research}
%
% % В начале раздела  можно напомнить его цель
%

\section{Почему мобильное приложение?}
\label{sec:whyApp}

Мобильные приложения могут быть инструментом для быстрой доставки информации, чего нельзя добиться при помощи обычного веб-приложения.

Индустрия мобильных устройств очень быстро развивается.
На смену старым устройствам приходят более новые, современные и обладающие большим спектром возможностей.
Количество пользователей с каждым годом растёт.
сейчас смартфоны есть почти у всех студентов и они редко с ними расстаются надолго.
Конечно, важную роль играет программная составляющая "--- мобильные приложения.
Они существуют совершенно разной направленности:
\begin{itemize}
  \item развлекательные (игры, музыкальные и видео проигрыватели и т.д.);
  \item коммуникационные (мессенджеры, навигаторы и т.д.);
  \item справочные (словари, базы знаний);
  \item прикладные (все остальные от графического редактора до калькулятора).
\end{itemize}

Кроме того, популярность мобильных приложений повлекла за собой появление мощных инструментов разработки, большого количества библиотек и фреймворков, что, в свою очередь, сделало разработку приложений быстрой, лёгкой и продуктивной.


\section{Обзор аналогичных решений}
\label{sec:analogs}
Был произведен поиск существующих решений в Windows Phone Store, Google Play, Apple App Store, на GitHub, и было найдено 3 аналога.
К ним можно, так же, добавить сам сайт ОРИОКС.

Рассмотрим преимущества и недостатки каждого решения по отдельности.
В качестве критериев будем брать:
\begin{itemize}
  \item способ получения данных;
  \item возможность push-уведомлений;
  \item возможность просмотра расписания;
  \item возможность просмотра текущих предметов;
  \item возможность просмотра успеваемости;
  \item возможность просмотра списка долгов;
  \item возможность просмотра списка пересдач;
  \item наличие графического интерфейса;
  \item оффлайн доступ.
\end{itemize}
\Define{Push-уведомления}{способ рассылки уведомлений. У конечного пользователя такие уведомления выглядят как небольшое всплывающее окошко на экране телефона или в браузере}

\begin{figure}[ht]
  \centering
  \includegraphics[width=\textwidth/2]{inc/img/orioks_mobile.png}
  \caption{Главная страница ОРИОКС с открытым меню}
  \label{fig:orioksMobile}
\end{figure}

\subsection{Сайт ОРИОКС}
\label{subsec:orioks}

\Define{Платформа}{среда выполнения, в которой должен выполняться фрагмент программного обеспечения или объектный модуль с учётом накладываемых средой ограничений и предоставляемых возможностей}
\Define{Браузер}{программное обеспечение для просмотра информации из сети}
Платформа: Браузер.

\Abbrev{БД}{база данных}
Из плюсов: cайт ОРИОКС получает информацию напрямую из БД.
Это позволяет позволяет запрашивать только ту информацию, которая нужна в данный момент для отображения страницы.
Можно, так же, отметить наличие всех перечисленных выше возможностей, за исключением просмотра расписания (но его можно посмотреть на сайте МИЭТ)~\cite{orioks}.
Графический интерфейс присутствует, но не полностью адаптирован для мобильных устройств, то есть в некоторых местах элементы интерфейса не помещаются на экране, а в других "--- наоборот слишком много неиспользованного места (см.~рис.~\ref{fig:orioksMobile}).

\Abbrev{HTML}{hypertext markup language "--- язык гипертекстовой разметки}
К минусам можно отнести отсутствие push-уведомлений, невозможность просмотра информации без интернет соединения и необходимость загружать таблицы стилей и HTML разметку для просмотра страницы.

\begin{figure}[ht]
  \centering
  \includegraphics[width=\textwidth/2]{inc/img/miet_schedule.png}
  \caption{Экран расписания в приложении ``Расписание для МИЭТ''}
  \label{fig:mietSchedule}
\end{figure}

\subsection{Приложение ``Расписание для МИЭТ''}
\label{subsec:appMietSchedule}
Платформа: Android.
Дополнительная информация: около тысячи установок;
рейтинг на Google Play "--- 4.7 из 5;
дата последнего обновления "--- 22.04.2018.

Приложение предназначено для просмотра новостей МИЭТ и расписания любой группы с возможностью скачать его и использовать в оффлайн-режиме.
Раньше присутствовал функционал просмотра успеваемости, но из-за изменений на сайте ОРИОКС этот функционал стал недоступен~\cite{market:mietSchedule}.

Получение расписания реализовано через API, это плюс.
Для получения текущей успеваемости использовался синтаксический анализ сайта.
При таком подходе любое изменение в таблице стилей или HTML разметке сайта приводит к неработоспособности приложения, что и случилось.

Из требуемых возможностей присутствует просмотр и кэширование расписания, это позволяет просматривать его без интернет-соединения
Просмотр текущих предметов, успеваемости, списка долгов и пересдач невозможен.

Графический интерфейс присутствует, но не соответствует требованиям Material Design.
Например, слишком маленькие отступы от краёв экрана (см.~рис.~\ref{fig:mietSchedule}).

\begin{figure}[ht]
  \centering
  \includegraphics[width=\textwidth/2]{inc/img/orioks_live.png}
  \caption{Главный экран с открытым меню в приложении ``Ориокс Live''}
  \label{fig:orioksLive}
\end{figure}

\subsection{Приложение ``ОРИОКС Live''}
\label{subsec:appOrioksLive}
Платформа: Android.
Дополнительная информация: около тысячи установок;
рейтинг на Google Play "--- 3.9 из 5;
дата последнего обновления "--- 30.09.2015.

Приложение предназначено для просмотра успеваемости, списка контрольных мероприятий и информации о преподавателях~\cite{market:orioksLive}.

Данные получаются при помощи непубличного программного интерфейса, но интерфейс был удалён, т.к. был сделан неофициально и работа с ОРИОКС стала невозможна.
Приложение не обновлялось с 2015 года и на данный момент не работает.

Заявленный функционал проверить не удалось из-за невозможности авторизации, поэтому все эти возможности отметим как отсутствующие.
Рабочей осталась только возможность просмотра новостей.

Графический интерфейс присутствует, но не соответствует требованиям Material Design.
Неправильные отступы, слишком контрастные цвета (см.~рис.~\ref{fig:orioksLive}).

\begin{figure}[ht]
  \centering
  \includegraphics[width=\textwidth/2]{inc/img/open_orioks.jpg}
  \caption{Интерфейс управления Telegram-ботом ``Open Orioks''}
  \label{fig:openOrioks}
\end{figure}

\subsection{Telegram-бот ``Open Orioks''}
\label{subsec:botOpenOrioks}
Платформа: Telegram.
Дополнительная информация: последнее обновление "--- 12.10.2017.

Бот позволяет просматривать расписание на день, текущую успеваемость и список контрольных мероприятий по каждому предмету.

Для получения данных используется синтаксический анализ (как и в пункте~\ref{subsec:appMietSchedule}), но за счёт того, что данные всех студентов обновляются раз в полчаса и сохраняются в хранилище бота, скорость получения данных конечным пользователем сравнима со скоростью получения данных из БД~\cite{github:openOrioks}.

Собственного графического интерфейса нет.
Взаимодействие с ботом производится через текстовые сообщения в мессенджере Telegram.
Использование при отсутствии интернета невозможно, но можно просматривать предыдущие ответы бота, что можно считать частичным кэшированием.

\subsection{Итоги}
\label{subsec:analogsSummary}

По итогом обзора аналогичных решений была построена таблица~\ref{tab:analogs}.
Как видно из таблицы, нет ни одного решения, которое бы соответствовало всем необходимым параметрам, что еще раз доказывает актуальность задачи.

\begin{table}[ht]
  \caption{Сравнение аналогичных решений}
  \label{tab:analogs}
  \small
  \begin{tabularx}{\textwidth}{|>{\raggedright}m{.203\textwidth}|>{\hsize=.9\hsize}C|>{\hsize=1.2\hsize}C|C|>{\hsize=.9\hsize}C|c|}
    \hline
    Критерий & Сайт ОРИОКС~\cite{orioks} & Приложение ``Расписание для МИЭТ''~\cite{market:mietSchedule} & Приложение ``ОРИОКС Live''~\cite{market:orioksLive} & Бот ``Open Orioks''~\cite{github:openOrioks} & МП СУПС \\
    \hline
    Скорость получения информации & Средняя & Высокая & Низкая  & Средняя & Высокая \\
    \hline
    Push-уведомления              & $-$     & $-$     & $-$     & $+$     & $+$     \\
    \hline
    Расписание                    & $\pm$   & $+$     & $-$     & $\pm$   & $+$     \\
    \hline
    Текущие предметы              & $+$     & $-$     & $-$     & $+$     & $+$     \\
    \hline
    Успеваемость                  & $+$     & $-$     & $-$     & $+$     & $+$     \\
    \hline
    Долги                         & $+$     & $-$     & $-$     & $-$     & $+$     \\
    \hline
    Пересдачи                     & $+$     & $-$     & $-$     & $-$     & $+$     \\
    \hline
    Графический интерфейс         & $+$     & $-$     & $-$     & $-$     & $+$     \\
    \hline
    Оффлайн доступ                & $-$     & $+$     & $-$     & $\pm$   & $+$     \\
    \hline
  \end{tabularx}
  \caption*{
    \small
    \raggedright
    $+$ -- указанная возможность присутствует\\
    $\pm$ -- указанная возможность частично присутствует\\
    $-$ -- указанная возможность отсутствует
  }
\end{table}

В приложении ``Расписание для МИЭТ'' есть кэширование, но работает оно только для расписания, т.к. остальные функции недоступны.
Только Telegram-бот ``Open Orioks'' предоставляет возможность push-уведомлений, но не обладает всеми требуемыми возможностями, т.к. при увеличении количества выполняемых задач, бот становится неудобен в использовании.

\section{Обзор популярных мобильных платформ}
\label{sec:platforms}

\section{Исследование структуры ОРИОКС}
\label{sec:orioksStructure}

\section{Входные и выходные данные}
\label{sec:io}

\section{Постановка задачи}
\label{sec:problem}
