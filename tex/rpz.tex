%% Преамбула TeX-файла

% 1. Стиль и язык
\documentclass[liberation]{G7-32}

% Остальные стандартные настройки убраны в preamble.inc.tex.
\sloppy

% Настройки стиля ГОСТ 7-32
% Для начала определяем, хотим мы или нет, чтобы рисунки и таблицы нумеровались в пределах раздела, или нам нужна сквозная нумерация.
\EqInChapter % формулы будут нумероваться в пределах раздела
\TableInChapter % таблицы будут нумероваться в пределах раздела
\PicInChapter % рисунки будут нумероваться в пределах раздела

% Добавляем гипертекстовое оглавление в PDF
\usepackage[
bookmarks=true, colorlinks=true, unicode=true,
urlcolor=black,linkcolor=black, anchorcolor=black,
citecolor=black, menucolor=black, filecolor=black,
]{hyperref}

\AfterHyperrefFix

\usepackage{microtype}% полезный пакет для микротипографии, увы под xelatex мало чего умеет, но под pdflatex хорошо улучшает читаемость

% Тире могут быть невидимы в Adobe Reader
\ifInvisibleDashes
\MakeDashesBold
\fi

\usepackage{graphicx}   % Пакет для включения рисунков

% С такими оно полями оно работает по-умолчанию:
% \RequirePackage[left=20mm,right=10mm,top=20mm,bottom=20mm,headsep=0pt,includefoot]{geometry}
% Если вас тошнит от поля в 10мм --- увеличивайте до 20-ти, ну и про переплёт не забывайте
\geometry{ignorefoot}% считать от нижней границы текста


% Пакет Tikz
\usepackage{tikz}
\usetikzlibrary{arrows,positioning,shadows}

% Произвольная нумерация списков.
\usepackage{enumerate}

% ячейки в несколько строчек
\usepackage{multirow}

% itemize внутри tabular и выравнивание таблиц
\usepackage{paralist,array,tabularx}

% Немного снимаем боль при работе с таблицами, которые надо центрировать
\newcolumntype{L}{>{\raggedright\arraybackslash}l}
\newcolumntype{P}[1]{>{\centering\arraybackslash}p{#1}}
\newcolumntype{M}[1]{>{\centering\arraybackslash}m{#1}}

% tabularx
\newcolumntype{C}{>{\centering\arraybackslash}X}

%\setlength{\parskip}{1ex plus0.5ex minus0.5ex} % разрыв между абзацами
\setlength{\parskip}{0ex} % разрыв между абзацами
\usepackage{blindtext}

% Центрирование подписей к плавающим окружениям
\usepackage[justification=centering]{caption}

% Отступ после листинга
\captionsetup[listing]{belowskip=20pt}


% Настройки листингов.
\usepackage{local-minted}

% Полезные макросы листингов.
% Любимые команды
\newcommand{\Code}[1]{\textbf{#1}}
\newcommand{\todo}[1]{{\color{red}TODO: {#1}}}
\newcommand{\conclusions}{
  \backmatter
  \section{Выводы}
  \mainmatter
}


% Стиль титульного листа и заголовки
%\NirEkz{Экз. 3}             % Раскоментировать если не требуется
%\NirGrif{Секретно}                % Наименование грифа

%\gosttitle{Gost7-32}       % Шаблон титульной страницы, по умолчанию будет ГОСТ 7.32-2001, 
% Варианты GostRV15-110 или Gost7-32 

\NirOrgLongName{\small
МИНОБРНАУКИ РОССИИ\\
Федеральное государственное автономное образовательное учреждение высшего образования\\
<<Национальный исследовательский университет\\<<Московский институт электронной техники>>\\
~\\
Кафедра информатики и программного обеспечения вычислительных систем
}                   %% Полное название организации

%\NirUdk{УДК № 378.14}
%\NirGosNo{№ госрегистрации 01970006723}
%\NirInventarNo{Инв. № ??????}

%\NirConfirm{Согласовано}                  % Смена УТВЕРЖДАЮ
\NirBossStamp{}


\NirReportName{Фаткуллин Осип Андреевич}   % Можно поменять тип отчета
\NirAbout{} %Можно изменить о чем отчет

%\NirPartNum{Часть}{1}                      % Часть номер

\NirBareSubject{}                  % Убирает по теме если раскоментить

% \NirIsAnnotacion{АННОТАЦИОННЫЙ }         %% Раскомментируйте, если это аннотационный отчёт
%\NirStage{промежуточный}{Этап \No 1}{} %%% Этап НИР: {номер этапа}{вид отчёта - промежуточный или заключительный}{название этапа}
%\NirStage{}{}{} %%% Этап НИР: {номер этапа}{вид отчёта - промежуточный или 

\Nir{Бакалаврская работа\\по направлению 09.03.04 <<Программная инженерия>>}

\NirSubject{Разработка мобильного приложения сопровождения учебного\\
процесса студентов в системе ОРИОКС} % Наименование темы
%\NirFinal{}                        % Заключительный, если закоментировать то промежуточный
%\finalname{итоговый}               % Название финального отчета (Заключительный) 
%\NirCode{Шифр\,---\,САПР-РЛС-ФИЗТЕХ-1} % Можно задать шифр как в ГОСТ 15.110
\NirCode{Шифр МП СУПС}

\NirManager{Руководитель,\\доцент, к.п.н.}{Е.Л. Федотова~~~~} %% Название руководителя
\NirIsp{Cтудент}{О.А. Фаткуллин} %% Название руководителя

%\NirYear{1999}%% если нужно поменять год отчёта; если закомментировано, ставится текущий год
\NirTown{Москва}                           %% город, в котором написан отчёт



\begin{document}

  \frontmatter % выключает нумерацию ВСЕГО; здесь начинаются ненумерованные главы: реферат, введение, глоссарий, сокращения и прочее.

  \maketitle %создает титульную страницу


  %\listoffigures                         % Список рисунков

  %\listoftables                          % Список таблиц

  %\NormRefs % Нормативные ссылки
  % Команды \breakingbeforechapters и \nonbreakingbeforechapters
  % управляют разрывом страницы перед главами.
  % По-умолчанию страница разрывается.

  % \nobreakingbeforechapters
  % \breakingbeforechapters

  \tableofcontents

  \printnomenclature % Автоматический список сокращений

  \Introduction
\Abbrev{МП СУПС}{мобильное приложение сопровождения учебного процесса студентов в системе ОРОИКС}
\hyphenation{МИЭТе}
В современных учебных заведениях всё чаще практикуют использование систем сопровождения учебного процесса, которые позволяют учащимся видеть свои оценки, получать домашние задания, узнавать о событиях учебного процесса.
\Abbrev{ОРИОКС}{организация распределенного информационного обмена в корпоративных средах}
В МИЭТе такой системой является ОРИОКС (Организация Распределенного Информационного Обмена в Корпоративных Средах).

Слабым местом ОРИОКС является невозможность оперативного оповещения учащихся о событиях учебного процесса.
\Define{Веб-приложение}{клиент-серверное приложение, в котором клиент взаимодействует с сервером при помощи браузера, а за сервер отвечает — веб-сервер}
Это обусловлено тем, что ОРИОКС представляет из себя веб-приложение и подразумевает регулярное посещение через браузер.
Если же студент долгое время не заходит в систему, что бывает довольно часто, то он может пропустить информацию о важных событиях, например о пересдачах.
Так же, в ОРИОКС невозможна установка напоминаний о предстоящих событиях.

Еще одна проблема веб-приложений – неудобство использования на мобильных
устройствах.
Большая часть студентов посещает ОРИОКС именно с мобильных
устройств, и, хотя, дизайн сайта оптимизирован для работы на небольших экранах, всё же мобильные устройства накладывают свои ограничения.
\Abbrev{ПК}{персональный компьютер}
Скорость интернет соединения, чаще всего, ниже чем на стационарных ПК, кроме того, соединение может часто разрываться, что приводит к дискомфорту
при использовании веб-приложения, т.к.\ для каждой страницы браузер должен
загрузить, помимо данных, разметку и таблицу стилей.


\Define{Мобильное приложение}{программное приложение, предназначенное для работы на смартфонах или планшетных компьютерах}
\Abbrev{API}{application programming interface "--- внешний интерфейс взаимодействия с приложением}
Все эти недостатки можно устранить, создав мобильное приложение, которое будет получать данные от сервера ОРИОКС через API).
Это позволит запрашивать и получать только ту информацию, которая нужна для работы приложения в конкретный момент времени, так как приложению не нужно скачивать таблицу стилей и разметку.
В виду малого потребления трафика приложение сможет работать на смартфоне в фоновом режиме и отображать уведомления и напоминания в тот момент когда они только пришли.

Удобство работы с системой ОРИОКС при нестабильном интернет соединении тоже повысится за счёт меньшего объёма пересылаемых данных.
\Define{Кэширование}{один из способов оптимизации приложений. Заключается в том, чтобы сохранять на некоторое время и переиспользовать результаты медленных операций.}
Кроме того, приложение может кэшировать данные и использовать их даже при отсутствии интернет соединения.

На данный момент нет ни одного приложения работающего с ОРИОКС через API, обладающего полным функционалом и предоставляющего возможности push-уведомлений, поэтому задача является актуальной.
Целью данной работы является создание такого приложения, чтобы повысить оперативность оповещения студентов о событиях учебного процесса.

Задачи выпускной квалификационной работы:
\begin{itemize}
  \item сравнительный анализ существующих программных решений;
  \item выбор платформы;
  \item выбор языка и среды программирования;
  \item разработка алгоритма;
  \item разработка схемы данных;
  \item разработка пользовательского интерфейса;
  \item отладка и тестирование;
  \item разработка руководства оператора.
\end{itemize}

Пояснительная записка состоит из введения, исследовательского, конструкторского и технологического разделов, заключения, списка использованных источников и трёх приложений.

Исследовательский раздел содержит обзор существующих решений, исследование структуры ОРИОКС, описание входных и выходных данных, постановку целей и задач.
Конструкторский раздел содержит обзор и выбор языка программирования, среды разработки, выбор стека технологий, описание алгоритма работы, схему данных и макеты пользовательского интерфейса.
Технологический раздел включает в себя описания применяемых технологий, нюансы разработки под Android, описание процессов сборки, публикации, тестирования, отладки и сопровождения.

В приложении~\ref{ch:appendix1} содержится техническое задание на разработку МП СУПС.
Приложение~\ref{ch:appendix2} содержит фрагменты исходного кода приложения.
В приложении~\ref{ch:appendix3} "--- руководство оператора.


  \mainmatter % это включает нумерацию глав и секций в документе ниже

  \chapter{Исследовательский раздел}
\label{ch:research}
%
% % В начале раздела  можно напомнить его цель
%

\section{Почему мобильное приложение?}
\label{sec:whyApp}

Мобильные приложения могут быть инструментом для быстрой доставки информации, чего нельзя добиться при помощи обычного веб-приложения.

Индустрия мобильных устройств очень быстро развивается.
На смену старым устройствам приходят более новые, современные и обладающие большим спектром возможностей.
Количество пользователей с каждым годом растёт.
сейчас смартфоны есть почти у всех студентов и они редко с ними расстаются надолго.
Конечно, важную роль играет программная составляющая "--- мобильные приложения.
Они существуют совершенно разной направленности:
\begin{itemize}
  \item развлекательные (игры, музыкальные и видео проигрыватели и т.д.);
  \item коммуникационные (мессенджеры, навигаторы и т.д.);
  \item справочные (словари, базы знаний);
  \item прикладные (все остальные от графического редактора до калькулятора).
\end{itemize}

Кроме того, популярность мобильных приложений повлекла за собой появление мощных инструментов разработки, большого количества библиотек и фреймворков, что, в свою очередь, сделало разработку приложений быстрой, лёгкой и продуктивной.


\section{Обзор аналогичных решений}
\label{sec:analogs}
Был произведен поиск существующих решений в Windows Phone Store, Google Play, Apple App Store, на GitHub, и было найдено 3 аналога.
К ним можно, так же, добавить сам сайт ОРИОКС.

Рассмотрим преимущества и недостатки каждого решения по отдельности.
В качестве критериев будем брать:
\begin{itemize}
  \item способ получения данных;
  \item возможность push-уведомлений;
  \item возможность просмотра расписания;
  \item возможность просмотра текущих предметов;
  \item возможность просмотра успеваемости;
  \item возможность просмотра списка долгов;
  \item возможность просмотра списка пересдач;
  \item наличие графического интерфейса;
  \item оффлайн доступ.
\end{itemize}
\Define{Push-уведомления}{способ рассылки уведомлений. У конечного пользователя такие уведомления выглядят как небольшое всплывающее окошко на экране телефона или в браузере}

\begin{figure}[ht]
  \centering
  \includegraphics[width=\textwidth/2]{inc/img/orioks_mobile.png}
  \caption{Главная страница ОРИОКС с открытым меню}
  \label{fig:orioksMobile}
\end{figure}

\subsection{Сайт ОРИОКС}
\label{subsec:orioks}

\Define{Платформа}{среда выполнения, в которой должен выполняться фрагмент программного обеспечения или объектный модуль с учётом накладываемых средой ограничений и предоставляемых возможностей}
\Define{Браузер}{программное обеспечение для просмотра информации из сети}
Платформа: Браузер.

\Abbrev{БД}{база данных}
Из плюсов: cайт ОРИОКС получает информацию напрямую из БД.
Это позволяет позволяет запрашивать только ту информацию, которая нужна в данный момент для отображения страницы.
Можно, так же, отметить наличие всех перечисленных выше возможностей, за исключением просмотра расписания (но его можно посмотреть на сайте МИЭТ)~\cite{orioks}.
Графический интерфейс присутствует, но не полностью адаптирован для мобильных устройств, то есть в некоторых местах элементы интерфейса не помещаются на экране, а в других "--- наоборот слишком много неиспользованного места (см.~рис.~\ref{fig:orioksMobile}).

\Abbrev{HTML}{hypertext markup language "--- язык гипертекстовой разметки}
К минусам можно отнести отсутствие push-уведомлений, невозможность просмотра информации без интернет соединения и необходимость загружать таблицы стилей и HTML разметку для просмотра страницы.

\begin{figure}[ht]
  \centering
  \includegraphics[width=\textwidth/2]{inc/img/miet_schedule.png}
  \caption{Экран расписания в приложении ``Расписание для МИЭТ''}
  \label{fig:mietSchedule}
\end{figure}

\subsection{Приложение ``Расписание для МИЭТ''}
\label{subsec:appMietSchedule}
Платформа: Android.
Дополнительная информация: около тысячи установок;
рейтинг на Google Play "--- 4.7 из 5;
дата последнего обновления "--- 22.04.2018.

Приложение предназначено для просмотра новостей МИЭТ и расписания любой группы с возможностью скачать его и использовать в оффлайн-режиме.
Раньше присутствовал функционал просмотра успеваемости, но из-за изменений на сайте ОРИОКС этот функционал стал недоступен~\cite{market:mietSchedule}.

Получение расписания реализовано через API, это плюс.
Для получения текущей успеваемости использовался синтаксический анализ сайта.
При таком подходе любое изменение в таблице стилей или HTML разметке сайта приводит к неработоспособности приложения, что и случилось.

Из требуемых возможностей присутствует просмотр и кэширование расписания, это позволяет просматривать его без интернет-соединения
Просмотр текущих предметов, успеваемости, списка долгов и пересдач невозможен.

Графический интерфейс присутствует, но не соответствует требованиям Material Design.
Например, слишком маленькие отступы от краёв экрана (см.~рис.~\ref{fig:mietSchedule}).

\begin{figure}[ht]
  \centering
  \includegraphics[width=\textwidth/2]{inc/img/orioks_live.png}
  \caption{Главный экран с открытым меню в приложении ``Ориокс Live''}
  \label{fig:orioksLive}
\end{figure}

\subsection{Приложение ``ОРИОКС Live''}
\label{subsec:appOrioksLive}
Платформа: Android.
Дополнительная информация: около тысячи установок;
рейтинг на Google Play "--- 3.9 из 5;
дата последнего обновления "--- 30.09.2015.

Приложение предназначено для просмотра успеваемости, списка контрольных мероприятий и информации о преподавателях~\cite{market:orioksLive}.

Данные получаются при помощи непубличного программного интерфейса, но интерфейс был удалён, т.к. был сделан неофициально и работа с ОРИОКС стала невозможна.
Приложение не обновлялось с 2015 года и на данный момент не работает.

Заявленный функционал проверить не удалось из-за невозможности авторизации, поэтому все эти возможности отметим как отсутствующие.
Рабочей осталась только возможность просмотра новостей.

Графический интерфейс присутствует, но не соответствует требованиям Material Design.
Неправильные отступы, слишком контрастные цвета (см.~рис.~\ref{fig:orioksLive}).

\begin{figure}[ht]
  \centering
  \includegraphics[width=\textwidth/2]{inc/img/open_orioks.jpg}
  \caption{Интерфейс управления Telegram-ботом ``Open Orioks''}
  \label{fig:openOrioks}
\end{figure}

\subsection{Telegram-бот ``Open Orioks''}
\label{subsec:botOpenOrioks}
Платформа: Telegram.
Дополнительная информация: последнее обновление "--- 12.10.2017.

Бот позволяет просматривать расписание на день, текущую успеваемость и список контрольных мероприятий по каждому предмету.

Для получения данных используется синтаксический анализ (как и в пункте~\ref{subsec:appMietSchedule}), но за счёт того, что данные всех студентов обновляются раз в полчаса и сохраняются в хранилище бота, скорость получения данных конечным пользователем сравнима со скоростью получения данных из БД~\cite{github:openOrioks}.

Собственного графического интерфейса нет.
Взаимодействие с ботом производится через текстовые сообщения в мессенджере Telegram.
Использование при отсутствии интернета невозможно, но можно просматривать предыдущие ответы бота, что можно считать частичным кэшированием.

\subsection{Итоги}
\label{subsec:analogsSummary}

По итогом обзора аналогичных решений была построена таблица~\ref{tab:analogs}.
Как видно из таблицы, нет ни одного решения, которое бы соответствовало всем необходимым параметрам, что еще раз доказывает актуальность задачи.

\begin{table}[ht]
  \caption{Сравнение аналогичных решений}
  \label{tab:analogs}
  \small
  \begin{tabularx}{\textwidth}{|>{\raggedright}m{.203\textwidth}|>{\hsize=.9\hsize}C|>{\hsize=1.2\hsize}C|C|>{\hsize=.9\hsize}C|c|}
    \hline
    Критерий & Сайт ОРИОКС~\cite{orioks} & Приложение ``Расписание для МИЭТ''~\cite{market:mietSchedule} & Приложение ``ОРИОКС Live''~\cite{market:orioksLive} & Бот ``Open Orioks''~\cite{github:openOrioks} & МП СУПС \\
    \hline
    Скорость получения информации & Средняя & Высокая & Низкая  & Средняя & Высокая \\
    \hline
    Push-уведомления              & $-$     & $-$     & $-$     & $+$     & $+$     \\
    \hline
    Расписание                    & $\pm$   & $+$     & $-$     & $\pm$   & $+$     \\
    \hline
    Текущие предметы              & $+$     & $-$     & $-$     & $+$     & $+$     \\
    \hline
    Успеваемость                  & $+$     & $-$     & $-$     & $+$     & $+$     \\
    \hline
    Долги                         & $+$     & $-$     & $-$     & $-$     & $+$     \\
    \hline
    Пересдачи                     & $+$     & $-$     & $-$     & $-$     & $+$     \\
    \hline
    Графический интерфейс         & $+$     & $-$     & $-$     & $-$     & $+$     \\
    \hline
    Оффлайн доступ                & $-$     & $+$     & $-$     & $\pm$   & $+$     \\
    \hline
  \end{tabularx}
  \caption*{
    \small
    \raggedright
    $+$ -- указанная возможность присутствует\\
    $\pm$ -- указанная возможность частично присутствует\\
    $-$ -- указанная возможность отсутствует
  }
\end{table}

В приложении ``Расписание для МИЭТ'' есть кэширование, но работает оно только для расписания, т.к. остальные функции недоступны.
Только Telegram-бот ``Open Orioks'' предоставляет возможность push-уведомлений, но не обладает всеми требуемыми возможностями, т.к. при увеличении количества выполняемых задач, бот становится неудобен в использовании.

\section{Обзор популярных мобильных платформ}
\label{sec:platforms}

\section{Исследование структуры ОРИОКС}
\label{sec:orioksStructure}

\section{Входные и выходные данные}
\label{sec:io}

\section{Постановка задачи}
\label{sec:problem}

  \chapter{Конструкторский раздел}
\label{ch:design}

\section{Выбор языка программирования}
\label{sec:language}

\section{Выбор среды разработки}
\label{sec:ide}

\section{Выбор стека технологий}
\label{sec:stack}

\section{Архитектура и алгоритм работы МП СУПС}
\label{sec:architecture}

\section{Разработка пользовательского интерфейса}
\label{sec:gui}

\conclusions
\label{sec:designConclusions}

  \chapter{Технологический раздел}
\label{ch:tech}


\section{Разработка под Android}
\label{sec:dev}

\subsection{Система контроля версий}
\label{subsec:vcs}

\subsection{Структура проекта}
\label{subsec:arch}

\subsection{Стиль кода}
\label{subsec:codestyle}

\subsection{Разработка пользовательского интерфейса}
\label{subsec:ui}

\subsection{Работа с данными}
\label{subsec:data}

\subsection{Безопасность}
\label{subsec:security}

\subsection{Многопоточность и работа с сетью}
\label{subsec:async}


\section{Сборка приложения}
\label{sec:build}

\subsection{Система автоматической сборки Gradle}
\label{subsec:gradle}

\subsection{Подключение зависимостей}
\label{subsec:libs}

\subsection{Подпись приложения}
\label{subsec:signing}

\subsection{Публикаця приложения}
\label{subsec:publish}

\subsection{Непрерывная интеграция}
\label{subsec:ci}


\section{Тестирование и отладка в Android разработке}
\label{sec:testing}

\subsection{Методы тестирования программного обеспечения}
\label{subsec:testing:methods}

\subsection{Методология разработки TDD}
\label{subsec:testing:tdd}

\subsection{Тестирование пользовательского интерфейса}
\label{subsec:testing:ui}

\subsection{Отладка в IntelliJ IDEA}
\label{subsec:debug}

\conclusions
\label{sec:techConclusions}


  \backmatter %% Здесь заканчивается нумерованная часть документа и начинаются ссылки и

  \Conclusion % заключение к отчёту

В результате проделанной работы стало ясно, что ничего не ясно\ldots

%%% Local Variables: 
%%% mode: latex
%%% TeX-master: "rpz"
%%% End: 
 %% заключение


  %
% Список литературы при помощи BibTeX
%

\bibliographystyle{ugost2008}
\bibliography{rpz}



  \appendix % Тут идут приложения

  \chapter{Техническое задание}
\label{ch:appendix1}

  \chapter{Руководство оператора}
\label{ch:appendix2}

  \chapter{Текст программы}
\label{ch:appendix3}
\setcounter{page}{1}
\\
\begingroup
  \kotlinfile{inc/src/code/main.gradle.kts}
  \captionof{listing}{Сценарий сборки корневого проекта (build.gradle.kts)}
\endgroup
\\
\begingroup
  \kotlinfile{inc/src/code/build.gradle.kts}
  \captionof{listing}{Сценарий сборки модуля presentation (presentation/build.gradle.kts)}
\endgroup
\\
\begingroup
  \kotlinfile{inc/src/code/dependencies.gradle.kts}
  \captionof{listing}{Файл, где производится настройка зависимостей (presentation/build.gradle.kts)}
\\
\begingroup
  \kotlinfile{inc/src/code/AuthFragment.kt}
  \captionof{listing}{Фрагмент экрана авторизации (presentation/auth/fragment/AuthFragment.kt)}
\endgroup
\\
\begingroup
  \kotlinfile{inc/src/code/AuthPresenter.kt}
  \captionof{listing}{Представитель экрана авторизации (presentation/auth/presenter/AuthPresenter.kt)}
\endgroup
\\
\begingroup
\kotlinfile{inc/src/code/Validator.kt}
\captionof{listing}{Утилитарный класс для валидации полей (domain/util/Validator.kt)}
\endgroup
\\
\begingroup
  \kotlinfile{inc/src/code/AuthView.kt}
  \captionof{listing}{Абстракция View экрана авторизации (presentation/auth/view/AuthView.kt)}
\endgroup
\\
\begingroup
  \xmlfile{inc/src/code/screen_auth.xml}
  \captionof{listing}{Макет экрана авторизации (res/layout/screen_auth.xml)}
\endgroup
\\
\begingroup
  \yamlfile{inc/src/code/.travis.yml}
  \captionof{listing}{Настройки Travis CI (.travis.yml)}
\endgroup


\end{document}
