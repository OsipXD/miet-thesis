\chapter{Введение}
\label{ch:intro}

Системы, изначально предназначенные для работы на ПК, требуют адаптации для мобильных устройств. К приложениям на мобильных устройствах применяются иные требования, которые зачастую являются более жесткими. На данный момент система ОРИОКС с мобильных устройств доступна только через браузер, и при таком способе доступа тратится больше трафика, чем при доступе из приложения (так как приложению не нужно скачивать таблицы стилей, HTML-разметку и рисунки).
Отсутствие приложения делает, невозможным доступ к системе ОРИОКС при отсутствии интернета или сильно его затрудняет плохом качестве соединения.
Кроме того, важной частью приложения является возможность своевременного оповещения студентов о событиях учебного процесса.

Требуется разработать новое приложение, принцип работы которого, будет совершенно отличаться от принципа работы уже существующих приложений и будет основан не на синтаксическом разборе (``парсинге") страниц, а на экономичных и быстрых запросах к публичному интерфейсу (далее API).
Приложение должно быть нацелено на студентов и иметь возможность быстрого оповещения студентов о событиях учебного процесса, иметь режим работы ``оффлайн" и обеспечивать быстрый доступ к информации.

Настоящее техническое задание определяет требования к мобильному приложению сопровождения учебного процесса студентов для системы ОРИОКС.
