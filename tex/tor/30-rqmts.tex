\vspace{0.75cm}
\chapter{Требования к программе}
\label{ch:rqmts}
Разрабатываемое ПО должно соответствовать структуре мобильной ОС Android.

\section{Требования к функциональным характеристикам}
\label{sec:funcRqmts}

\subsection{Состав выполняемых функций}
\label{subsec:functions}

Разрабатываемое ПО должно обеспечить выполнение следующих функций:
\begin{itemize}
  \item авторизация при помощи номера студенческого билета и пароля;
  \item отображение информации об успеваемости и текущих предметах;
  \item просмотр списка долгов и пересдач;
  \item получение информации по пересдачам и контрольным мероприятиям;
  \item уведомления о назначенных пересдачах, поставленных баллах и прочих событиях учебного процесса в режиме реального времени;
  \item получение новостей;
  \item расписание занятий;
  \item просмотр закэшированной информации в режиме “оффлайн”;
  \item отображение номера недели и числитель/знаменатель.
\end{itemize}

Желательные функции:
\begin{itemize}
  \item раздел FAQ;
  \item раздел портфолио;
  \item раздел ДЗ.
\end{itemize}

\subsection{Организация входных и выходных данных}
\label{subsec:io}
В качестве входных данных должен использоваться ответ от сервера ОРИОКСа в формате JSON. Для доступа к этим данным пользователь предварительно должен авторизоваться при помощи графического интерфейса.

На выходе пользователь получает адаптированное для экрана его смартфона отображение полученных с сервера данных.

\section{Требования к надежности}
\label{sec:reliabilityRqmts}
Разрабатываемое приложение должно получать с сервера информацию, к которой имеет доступ только авторизованный пользователь.
В связи с этим, одним из главных требований к надежности является обеспечение безопасной авторизации и безопасного хранения данных.
Для обеспечения надежности в ПО должно быть предусмотрено:
\begin{itemize}
  \item использование протокола HTTPS для соединение с сервером;
  \item проверка подлинности сертификата сервера;
  \item сохранение чувствительных данных в памяти устройства в зашифрованном виде;
  \item отображение сообщений об ошибках при неверно заданных данных;
  \item отображение сообщений об ошибках при сбоях запросов к серверу;
  \item обработка ``пустых состояний";
  \item валидация вводимых данных.
\end{itemize}

\section{Условия эксплуатации и требования к составу и параметрам технических средств}
\label{sec:expluatation}
Пользователи ПО должны иметь навыки работы на мобильных устройствах, а так же навыки работы с системой ОРИОКС.
Требования к составу и параметрам технических средств представлены в таблицах~\ref{tab:rqmtsMin} и~\ref{tab:rqmtsRecommended}.

\begin{table}[ht]
  \centering
  \caption{Минимальный состав технических средств и их технические характеристики}
  \label{tab:rqmtsMin}
  \begin{tabular}{|l|l|}
    \hline Архитектура процессора & ARM \\ \hline
    RAM (оперативная память) & 256 МБ \\ \hline
    OS (операционная система) & Android 4.4 \\ \hline
    HDD (объем свободного места на жестком диске) & 25 МБ \\ \hline
    Разрешающая способность экрана & \textasciitilde 160dpi (mdpi) \\ \hline
    Дополнительные требования & Интернет \\ \hline
  \end{tabular}
\end{table}

\begin{table}[ht]
  \centering
  \caption{Рекомендуемый состав технических средств и их технические характеристики}
  \label{tab:rqmtsRecommended}
  \begin{tabular}{|l|l|}
    \hline Архитектура процессора & ARM64 \\ \hline
    RAM (оперативная память) & 512 МБ \\ \hline
    OS (операционная система) & Android 7.1 \\ \hline
    HDD (объем свободного места на жестком диске) & 50 МБ \\ \hline
    Разрешающая способность экрана & \textasciitilde 320dpi (xhdpi) \\ \hline
    Дополнительные требования & Интернет \\ \hline
  \end{tabular}
\end{table}

\section{Требования к информационной и программной совместимости}
\label{sec:infoRqmts}
Приложение должно работать под операционными системами Android версии 4.4 и выше.

Метод решения задачи базируется на стандартных приёмах работы с RESTful API\@.

Среда разработки - IntelliJ IDEA 2018.

\section{Требования к транспортировке и хранению}
\label{sec:storageRqmts}
Передача информация должна производиться по шифрованному каналу (протокол HTTPS).
Приложение должно сохранять все данные в приватном хранилище устройства, а чувствительные данные дополнительно шифровать.
Приложение должно хранить только последнюю полученную версию данных.

\section{Специальные требования}
\label{sec:specialRqmts}
Не предъявляются.
