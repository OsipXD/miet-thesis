\vspace{0.75cm}
\chapter{Требования к программной документации}
\label{ch:docsRqmts}
Основными  документами,  регламентирующими  разработку  будущих  программ, должны  быть  документы  Единой  Системы  Программной  Документации  (ЕСПД): руководство оператора, описание применения.

\vspace{0.75cm}
\chapter{Технико-экономические показатели}
\label{ch:techEconomy}
Основным назначением разрабатываемого ПО является упрощение доступа студентов к системе ОРИОКС.
Подобные приложения позволяют получать уведомления в реальном времени, что способствует быстрому получению информации и как следствие к повышению эффективности учебного процесса.

\vspace{0.75cm}
\chapter{Стадии и этапы разработки}
\label{ch:stages}
График составлен  согласно  рабочему  плану студента  четвертого  курса  дневной  формы обучения.

\begin{center}
  \centering
  \begin{longtable}{| c | L{4.5cm}| c | c | L{4.5cm}|}
    \hline
    № п/п & Наименование работы &  Дата начала & Дата окончания & Форма отчётности \\
    \hline
    1 & Постановка задачи & 07.02.2018 & 10.02.2018 & Эскиз слайда \\
    \hline
    2 & Исследование предметной области & 11.02.2018 & 12.02.2018 & Требования к МП СУПС \\
    \hline
    3 & Анализ существующих программных решений & 13.02.2018 & 14.02.2018 & Сравнительная таблица \\
    \hline
    4 & Разработка ТЗ & 15.02.2018 & 21.02.2018 & Утвержденное ТЗ \\
    \hline
    5 & Выбор используемых технологий & 22.02.2018 & 22.02.2018 & Язык и средства разработки \\
    \hline
    6 & Разработка предварительного алгоритма & 23.02.2018 & 25.02.2018 & Схема предварительного алгоритма \\
    \hline
    7 & Разработка предварительной структуры данных & 26.02.2018 & 28.02.2018 & Схема  предварительной структуры данных \\
    \hline
    8 & Создание прототипа пользовательского интерфейса & 01.03.2018 & 06.03.2018 & Эскиз экранов \\
    \hline
    9 & Подготовка отчёта по учебной практике & 07.03.2018 & 11.03.2018 & Отчёт по учебной практике \\
    \hline
    10 & Уточнение алгоритма & 12.03.2018 & 15.03.2018 & Уточнённая схема алгоритма \\
    \hline
    11 & Уточнение схемы данных & 15.03.2018 & 17.03.2018 & Уточненная схема данных\\
    \hline
    12 & Программирование и отладка & 17.03.2018 & 08.04.2018 & Тексты программ \\
    \hline
    13 & Разработка программных документов & 09.04.2018 & 17.04.2018 & Документация МП СУПС \\
    \hline
    14 & Проведение испытаний  & 18.04.2018 & 19.04.2018 & Результаты испытаний \\
    \hline
    15 & Корректировка программ и документов по результатам испытаний & 20.04.2018 & 04.05.2018 & Скорректированные тексты программ и документы \\
    \hline
    16 & Подготовка отчёта по производственной практике & 05.05.2018 & 07.05.2018 & Отчёт по производственной практике \\
    \hline
    17 & Разработка пояснительной записки к ВКР & 08.05.2018 & 24.05.2018 & Пояснительная записка \\
    \hline
    18 & Разработка руководства оператора & 25.05.2018 & 28.05.2018 & Руководство оператора \\
    \hline
    19 & Разработка иллюстраций & 29.05.2018 & 04.06.2018 & Презентация \\
    \hline
    20 & Подготовка к предзащите ВКР & 05.06.2018 & 07.06.2018 & Отчёт по преддипломной практике \\
    \hline
    21 & Предзащита ВКР & 08.06.2018 & 08.06.2018 & Допуск к защите ВКР и правки \\
    \hline
    22 & Внесение правок в ВКР & 09.06.2018 & 10.06.2018 & Исправленные документы \\
    \hline
    23 & Защита ВКР & 15.06.2018 & 15.06.2018 & Защита \\
    \hline
  \end{longtable}
\end{center}

\vspace{0.75cm}
\chapter{Порядок контроля и приемки}
\label{ch:controll}
Для всех модулей должны быть разработаны тесты и проведена отладка.
Для работы в целом должна быть разработана контрольно-демонстрационная задача.

При этом проверяется выполнение всех функций программы.\\

\noindent
\begin{flushright}
  Cтудент гр. МП-45: \rule{2.5cm}{0.4pt} / Фаткуллин О.А. / \\
  <<\rule{1cm}{0.4pt}>> \rule{2.5cm}{0.4pt} 2018 г.
\end{flushright}
