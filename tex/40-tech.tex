\chapter{Технологический раздел}
\label{ch:tech}
В данном разделе описываются использованные технологии разработки, сборки и публикации, отладки и тестирования Android-приложений.

\section{Разработка под Android}
\label{sec:dev}

\subsection{Система контроля версий}
\label{subsec:vcs}

\subsection{Структура проекта}
\label{subsec:arch}

\subsection{Стиль кода}
\label{subsec:codestyle}

\subsection{Разработка пользовательского интерфейса}
\label{subsec:ui}

\subsection{Работа с данными}
\label{subsec:data}

\subsection{Безопасность}
\label{subsec:security}

\subsection{Многопоточность и работа с сетью}
\label{subsec:async}


\section{Сборка и публикация приложения}
\label{sec:build}

\subsection{Система автоматической сборки Gradle}
\label{subsec:gradle}

\Abbrev{DSL}{domain-specific language "--- язык, специфический для предметной области}
\Abbrev{XML}{extensible markup language "--- расширяемый язык разметки}
Для сборки приложения в APK файл при Android-разработке обычно используется Gradle.
Это система автоматизации сборки с открытым исходным кодом, ориентированная на гибкость и производительность.
В отличие от аналогов (Maven и Ant), сценарии сборки пишутся на Groovy или Kotlin DSL, а не на XML\@.
Это позволяет максимально снизить порог вхождения для программиста.

Gradle является официальным инструментом сборки для Android и поддерживает многие популярные языки и технологии.
Время сборки сильно уменьшается за счёт того, что при сборке используются результаты прошлых сборок, выполняется сборка только измененных файлов (инкрементальная сборка) и независимые между собой задачи выполняются параллельно.
Gradle поддерживается всеми IDE для Android-разработки, в том числе IntelliJ IDEA\@.

При создании нового проекта в IDEA можно выбрать какой DSL использовать для сценариев сборки.
В процессе разработки МП СУПС используется Kotlin DSL, т.к. язык проекта "--- Kotlin.
После указания основных настроек (название приложения, домен, целевая версия и т.д.) создаются стандартные сценарии сборки для всего проекта и для модуля 'app', в котором содержится код приложения.
Это обусловлено тем, что в проекте может быть несколько модулей, например, модули data, domain и presentation при использовании чистой архитектуры.
Gradle и Kotlin DSL обновляются чаще чем инструменты разработки Android, поэтому созданные по умолчанию сценарии сборки могут не работать в силу того, что Kotlin DSL находится в стадии активной разработки и не гарантирует обратную совместимость, а шаблоны рассчитаны на старые версии.

Помимо сценариев сборки в Gradle есть еще некоторые файлы настроек:
\begin{itemize}
  \item \code{gradle.properties} содержит настройки Gradle (здесь, например, можно выставить ограничение потребления памяти);
  \item \code{settings.gradle.kts} содержит список модулей, которые нужно подключить;
  \item \code{gradle/wrapper/gradle-wrapper.properties} содержит настройки обёртки Gradle (Gradle wrapper).
\end{itemize}

Gradle позволяет дописывать собственные вспомогательные классы, которые можно будет использовать при написании сценария сборки.
Такие классы помещают в модуль \code{buildSrc}, который подключён по умолчанию и этот собирается перед сборкой остальных модулей.
Еще одна возможность организации структуры сценариев сборки "--- разбивка сценария на несколько файлов, отвечающих за разные аспекты сборки.
Например, выделить отдельный файл, в котором будет собрана вся работа с зависимостями.
В Gradle для подключения внешних файлов сценариев предусмотрена функция:
\codeline{kotlin}{project.apply { from("path/to/script.gradle.kts") }}

Все необходимые задачи, такие как настройка проекта для работы с Android, сборка приложения в apk, подпись его ключом и пр.\ выделены в отдельный плагин "--- Android Gradle Plugin.
Подключение этого плагина осуществляется в сценарии уровня проекта в секции \code{buildscript.dependencies} (листинг~\ref{lst:rootBuildKts}).

\begin{listing}[H]
  \kotlinfile{inc/src/rootBuild.gradle.kts}
  \caption{Подключение Android Gradle Plugin версии 3.0.1}
  \label{lst:rootBuildKts}
\end{listing}

После того, как плагин подключен, в сценарии уровня модуля, нужно указать настройки сборки Android приложения, а именно, целевую версию, минимальную допустимую версию, версию инструментов сборки, ID и версию приложения (листинг~\ref{lst:moduleBuildKts}).

\begin{listing}[H]
  \kotlinfile{inc/src/moduleBuild.gradle.kts}
  \caption{Настройка сборки Android-приложения}
  \label{lst:moduleBuildKts}
\end{listing}

\subsection{Подключение зависимостей}
\label{subsec:libs}

\subsection{Подпись приложения}
\label{subsec:signing}

\subsection{Публикация приложения}
\label{subsec:publish}

\subsection{Непрерывная интеграция}
\label{subsec:ci}


\section{Тестирование и отладка в Android разработке}
\label{sec:testing}

\subsection{Методы тестирования программного обеспечения}
\label{subsec:testing:methods}

\subsection{Методология разработки TDD}
\label{subsec:testing:tdd}

\subsection{Тестирование пользовательского интерфейса}
\label{subsec:testing:ui}

\subsection{Отладка в IntelliJ IDEA}
\label{subsec:debug}

\conclusions
\label{sec:techConclusions}
