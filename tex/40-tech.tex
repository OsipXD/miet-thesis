\chapter{Технологический раздел}
\label{ch:tech}
В данном разделе описываются использованные технологии разработки, сборки и публикации, отладки и тестирования Android-приложений.

\section{Разработка под Android}
\label{sec:dev}

\subsection{Система контроля версий}
\label{subsec:vcs}

\subsection{Структура проекта}
\label{subsec:arch}

\subsection{Стиль кода}
\label{subsec:codestyle}

\subsection{Разработка пользовательского интерфейса}
\label{subsec:ui}

\subsection{Работа с данными}
\label{subsec:data}

\subsection{Безопасность}
\label{subsec:security}

\subsection{Многопоточность и работа с сетью}
\label{subsec:async}


\section{Сборка и публикация приложения}
\label{sec:build}

\subsection{Система автоматической сборки Gradle}
\label{subsec:gradle}

\Abbrev{APK}{Android package kit}
\Abbrev{DSL}{domain-specific language "--- язык, специфический для предметной области}
\Abbrev{XML}{extensible markup language "--- расширяемый язык разметки}
Для сборки приложения в APK файл при Android-разработке обычно используется Gradle.
Это система автоматизации сборки с открытым исходным кодом, ориентированная на гибкость и производительность.
В отличие от аналогов (Maven и Ant), сценарии сборки пишутся на Groovy или Kotlin DSL, а не на XML\@.
Это позволяет максимально снизить порог вхождения для программиста.

Gradle является официальным инструментом сборки для Android и поддерживает многие популярные языки и технологии.
Время сборки сильно уменьшается за счёт того, что при сборке используются результаты прошлых сборок, выполняется сборка только измененных файлов (инкрементальная сборка) и независимые между собой задачи выполняются параллельно.
Gradle поддерживается всеми IDE для Android-разработки, в том числе IntelliJ IDEA\@.

При создании нового проекта в IDEA можно выбрать какой DSL использовать для сценариев сборки.
В процессе разработки МП СУПС используется Kotlin DSL, т.к. язык проекта "--- Kotlin.
После указания основных настроек (название приложения, домен, целевая версия и т.д.) создаются стандартные сценарии сборки для всего проекта и для модуля 'app', в котором содержится код приложения.
Это обусловлено тем, что в проекте может быть несколько модулей, например, модули data, domain и presentation при использовании чистой архитектуры.
Gradle и Kotlin DSL обновляются чаще чем инструменты разработки Android, поэтому созданные по умолчанию сценарии сборки могут не работать в силу того, что Kotlin DSL находится в стадии активной разработки и не гарантирует обратную совместимость, а шаблоны рассчитаны на старые версии.

Помимо сценариев сборки в Gradle есть еще файлы настроек:
\begin{itemize}
  \item \code{gradle.properties} содержит настройки Gradle (здесь, например, можно выставить ограничение потребления памяти);
  \item \code{settings.gradle.kts} содержит список модулей, которые нужно подключить;
  \item \code{gradle-wrapper.properties} (находится по пути \code{gradle/wrapper/}) содержит настройки обёртки Gradle (Gradle wrapper).
\end{itemize}

Gradle позволяет дописывать собственные вспомогательные классы, которые можно будет использовать при написании сценария сборки.
Такие классы помещают в модуль \code{buildSrc}, который подключён по умолчанию и этот собирается перед сборкой остальных модулей.
Еще одна возможность организации структуры сценариев сборки "--- разбивка сценария на несколько файлов, отвечающих за разные аспекты сборки.
Например, выделить отдельный файл, в котором будет собрана вся работа с зависимостями.
В Gradle для подключения внешних файлов сценариев предусмотрена функция:
\codeline{kotlin}{project.apply { from("path/to/script.gradle.kts") }}

Все необходимые задачи, такие как настройка проекта для работы с Android, сборка приложения в APK, подпись его ключом и пр.\ выделены в отдельный плагин "--- Android Gradle Plugin.
Подключение этого плагина осуществляется в сценарии уровня проекта в секции \code{buildscript.dependencies} (листинг~\ref{lst:rootBuildKts}).

\begin{listing}[H]
  \kotlinfile{inc/src/rootBuild.gradle.kts}
  \caption{Подключение Android Gradle Plugin версии 3.0.1}
  \label{lst:rootBuildKts}
\end{listing}

После того, как плагин подключен, в сценарии уровня модуля, нужно указать настройки сборки Android приложения, а именно, целевую версию, минимальную допустимую версию, версию инструментов сборки, ID и версию приложения (листинг~\ref{lst:moduleBuildKts}).

\begin{listing}[H]
  \kotlinfile{inc/src/moduleBuild.gradle.kts}
  \caption{Настройка сборки Android-приложения}
  \label{lst:moduleBuildKts}
\end{listing}

\subsection{Управление зависимостями}
\label{subsec:libs}

Ни одно Android-приложение не обходится без зависимостей.
Проект всегда использует бибилиотеки, а так же может быть разбит на отдельные модули, которые зависят друг от друга.
Управление зависимостями "--- это метод декларирования, разрешения и использования зависимостей, требуемых проектом в автоматическом режиме.

В Gradle есть инструменты для управления зависимостями, которые покрывают все типичные сценарии, встречающиеся в современных проектах по разработке ПО\@.

\todo{Картинка управления зависимостями в Gradle}

Например есть проект, написанный на Java.
В некоторых файлах используются классы из Google Guava (библиотека с открытым исходным кодом, предоставляющая множество полезных функций).
Кроме того, проекту нужен JUnit для компиляции и выполнения тестов.

Guava и JUnit представляют собой зависимости этого проекта.
Программист, при написании сценария сборки, может определять зависимости для разных областей (scopes), например, для компиляции всего кода или для компиляции и выполнения тестов.
В Gradle область действия зависимости называется ``конфигурацией''.

Обычно зависимости подключаются в виде модулей.
Необходимо указать Gradle, где найти эти модули, чтобы их можно было использовать во время сборки.
Место хранения модулей называется репозиторием.
После указания репозиториев для сборки, Gradle будет знать, как находить и получать модули.
Репозитории могут быть разных типов: локальный каталог и удаленный или локальный репозиторий.
Для обеспечения совместимости с популярными хранилищами модулей Gradle может использовать Maven и Ivy репозитории

Gradle определяет требуемые зависимости во время выполнения сценария сборки.
Зависимости могут быть загружены из удаленного репозитория, извлечены из локального каталога или могут потребовать сборки другого проекта при многопроектной конфигурации.
Этот процесс называется разрешением зависимостей.

После разрешения Gradle хранит файлы зависимости в локальном кэше, называемом кэшем зависимостей.
При повторной сборке будут повторно использованы файлы, хранящиеся в кэше, чтобы избежать ненужных сетевых вызовов и ускорить сборку.

Модули могут предоставлять метаданные, то есть данные, которые дополнительно описывают модуль, например, координаты для нахождения его в репозитории, актуальную версия модуля, информацию об авторах и т.д.
В рамках метаданных модуль может объявить, что для его работы необходимы другие модули, а Gradle автоматически разрешает эти дополнительные зависимости, которые называют ``транзитивными зависимостями''.

Проекты с большим количеством зависимостей могут пострадать от антипаттерна, называемого ``ад зависимостей'' (dependency hell).
Для борьбы с этим Gradle предоставляет инструментарий для визуализации и анализа графа зависимостей проекта.

\subsubsection*{Объявление зависимостей}

Типичным примером библиотеки в Java-проекте является JUnit 4 (в Kotlin-проекте эта библиотека может быть заменена модулем \code{kotlin-test-junit}).

\begin{listing}[H]
  \kotlinfile[escapeinside=||]{inc/src/junitDependency.gradle.kts}
  \caption{Подключение JUnit 4 в качестве зависимости}
  \label{lst:junitDependency}
\end{listing}

В листинге~\ref{lst:junitDependency} приведен пример кода, который подключает JUnit в проект как зависимость для сборки и выполнения тестов.
В строке~\ref{line:repo} подключается репозиторий Maven Central, в котором находится модуль JUnit.
В строке~\ref{line:junit} происходит объявление самого JUnit, для этого указываются координаты артефакта в формате \code{<group>:<module>:<version>}.
В начале строки указана область действия зависимости или, в терминологии Gradle, конфигурация "--- \code{testImplementation}.
Существует много различных конфигураций, но самые часто используемые это \code{implementation} и \code{testImplementation}.
В первом случае зависимость используется для компиляции всего проекта, во втором только для компиляции тестового кода.

Проекты могут использовать более агрессивный подход объявления зависимостей.
Например, подключать всегда последнюю версию модуля.
Динамическая версия позволяет подключить последнюю версию или последнюю подверсию версии модуля.
Стоит помнить, что использование динамических версий несет потенциальный риск нарушения сборки.
Код перестанет компилироваться как только новая версия библиотеки не будет обратно совместима с использованной до этого.
Чтобы минимизировать риск можно динамически разрешать только минорную или патч версию и только при условии, что разработчик подключаемого модуля использует семантическое версионирование.
Gradle проверяет наличие новой версии библиотеки по прохождении 24 часов.

\begin{listing}[H]
  \kotlinfile{inc/src/junitDependencyDynamic.gradle.kts}
  \caption{Подключение JUnit 4 в качестве зависимости c динамически разрешаемой версией}
  \label{lst:junitDependencyDynamic}
\end{listing}

В сценарии, показанном в листинге~\ref{lst:junitDependencyDynamic} версия \code{4.+} будет разрешена в версию \code{4.12}, но как только появится новая минорная или патч версия, будет использоваться она.

Бывает, что разработчики публикуют ``снимки'' (snapshots) библиотеки при разработке новой версии.
В таком случае версия библиотеки не меняется, но сама библиотека обновляется.
Для разрешения такой ситуации в Gradle достаточно добавить к версии постфикс \code{-SNAPSHOT}.
По умолчанию, наличие измененной версии проверяется раз в 24 часа.

\subsection{Подпись приложения}
\label{subsec:signing}

\subsection{Публикация приложения}
\label{subsec:publish}

\subsection{Непрерывная интеграция}
\label{subsec:ci}


\section{Тестирование и отладка в Android разработке}
\label{sec:testing}

\subsection{Методы тестирования программного обеспечения}
\label{subsec:testing:methods}

\subsection{Методология разработки TDD}
\label{subsec:testing:tdd}

\subsection{Тестирование пользовательского интерфейса}
\label{subsec:testing:ui}

\subsection{Отладка в IntelliJ IDEA}
\label{subsec:debug}

\conclusions
\label{sec:techConclusions}
