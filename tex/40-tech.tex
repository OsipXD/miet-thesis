\chapter{Технологический раздел}
\label{ch:tech}
В данном разделе описываются использованные технологии разработки, сборки и публикации, отладки и тестирования Android-приложений.

\section{Разработка под Android}
\label{sec:dev}

\subsection{Система контроля версий}
\label{subsec:vcs}

\subsection{Структура проекта}
\label{subsec:arch}

\subsection{Стиль кода}
\label{subsec:codestyle}

\subsection{Разработка пользовательского интерфейса}
\label{subsec:ui}

\subsection{Работа с данными}
\label{subsec:data}

\subsection{Безопасность}
\label{subsec:security}

\subsection{Многопоточность и работа с сетью}
\label{subsec:async}


\section{Сборка и публикация приложения}
\label{sec:build}

\subsection{Система автоматической сборки Gradle}
\label{subsec:gradle}

\subsection{Подключение зависимостей}
\label{subsec:libs}

\subsection{Подпись приложения}
\label{subsec:signing}

\subsection{Публикаця приложения}
\label{subsec:publish}

\subsection{Непрерывная интеграция}
\label{subsec:ci}


\section{Тестирование и отладка в Android разработке}
\label{sec:testing}

\subsection{Методы тестирования программного обеспечения}
\label{subsec:testing:methods}

\subsection{Методология разработки TDD}
\label{subsec:testing:tdd}

\subsection{Тестирование пользовательского интерфейса}
\label{subsec:testing:ui}

\subsection{Отладка в IntelliJ IDEA}
\label{subsec:debug}

\conclusions
\label{sec:techConclusions}
