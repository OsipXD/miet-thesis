\chapter{Входные и выходные данные}
\label{ch:io}

\section{Получение раздела ``Контакты''}
\label{sec:contacts}

Запрос: \code{GET api/v1/contacts}

Формат ответа:
\begin{listing}[H]
  \httpfile[firstline=3,lastline=12]{inc/src/api/contacts.http}
\end{listing}
\vspace{-0.75cm}

Описание полей:\\
\code{classrooom} "--- номер аудитории кафедры.\\
\code{department} "--- кафедра.\\
\code{email} "--- электронная почта кафедры.\\
\code{inner\_tel} "--- внутренний телефон кафедры.\\
\code{tel} "--- телефон кафедры.

Пример ответа:
\begin{listing}[H]
  \httpfile[firstline=14]{inc/src/api/contacts.http}
\end{listing}
\vspace{-0.75cm}

\section{Получение информации о текущей неделе}
\label{sec:faq}

Запрос: \code{GET api/v1/schedule}

Формат ответа:
\begin{listing}[H]
  \httpfile[firstline=3,lastline=7]{inc/src/api/schedule.http}
\end{listing}
\vspace{-0.75cm}

Описание полей:\\
\code{type} "--- тип недели.
Используется для получение расписания группы:
\begin{itemize}
  \item 1: 1 числитель
  \item 2: 1 знаменатель
  \item 3: 2 числитель
  \item 4: 2 знаменатель
\end{itemize}
\code{week} "--- номер недели.

Пример ответа:
\begin{listing}[H]
  \httpfile[firstline=9]{inc/src/api/schedule.http}
\end{listing}
\vspace{-0.75cm}

\section{Получение списка групп}
\label{sec:groups}

Запрос: \code{GET api/v1/schedule/groups}

Формат ответа:
\begin{listing}[H]
  \httpfile[firstline=3,lastline=9]{inc/src/api/groups.http}
\end{listing}
\vspace{-0.75cm}

Описание полей:\\
\code{id} "--- идентификатор группы.\\
\code{name} "--- имя учебной группы.

Пример ответа:
\begin{listing}[H]
  \httpfile[firstline=11]{inc/src/api/groups.http}
\end{listing}
\vspace{-0.75cm}

\section{Получение расписания конкретной группы}
\label{sec:group}

Запрос: \code{GET api/v1/schedule/groups/<id>/types/<type>}

Где:\\
\code{<id>} "--- идентификатор группы.\\
\code{<type>} "--- тип недели.

Формат ответа:
\begin{listing}[H]
  \httpfile[firstline=3,lastline=17]{inc/src/api/groupSchedule.http}
\end{listing}
\vspace{-0.75cm}

Описание полей:\\
\code{day} "--- номер дня недели.\\
\code{classroom} "--- номер аудитории.\\
\code{name} "--- название предмета.\\
\code{number} "--- номер пары.\\
\code{teacher} "--- ФИО преподавателя.

Пример ответа:
\begin{listing}[H]
  \httpfile[firstline=19]{inc/src/api/groupSchedule.http}
\end{listing}
\vspace{-0.75cm}

\section{Получение информации о студенте}
\label{sec:student}

Запрос: \code{GET api/v1/student}

Формат ответа:
\begin{listing}[H]
  \httpfile[firstline=3,lastline=12]{inc/src/api/student.http}
\end{listing}
\vspace{-0.75cm}

Описание полей:\\
\code{course} "--- курс, на котором сейчас находится студент.\\
\code{full\_name} "--- ФИО студента.\\
\code{group} "--- учебная группа.\\
\code{kaf} "--- кафедра студента.\\
\code{np} "--- направление подготовки.\\
\code{semester} "--- текущий семестр (первый или второй).\\
\code{up} "--- профиль обучения.

Пример ответа:
\begin{listing}[H]
  \httpfile[firstline=14]{inc/src/api/student.http}
\end{listing}
\vspace{-0.75cm}
