\chapter{Входные и выходные данные}
\label{ch:io}

Входные данные:
\begin{itemize}
  \item токен аутентификации или логин/пароль для получения токена;
  \item запрашиваемый ресурс (аутентификация, дисциплины и т.д.);
  \item имя и версия приложения + операционная система клиента;
  \item HTTP-заголовок Accept;
  \item тип HTTP-запроса (GET, POST);
  \item тело запроса (для создания новой сущности), если запрос имеет тип POST\@.
\end{itemize}

~\\
Выходные данные:
\begin{itemize}
  \item код ответа сервера;
  \item описание ошибки, если запрос некорректный;
  \item запрашиваемые данные (если запрос типа GET).
\end{itemize}

~\\
Формат запроса:
\begin{listing}[H]
  \httpfile{inc/src/api/request.http}
\end{listing}
\vspace{-0.75cm}

Значение параметров:\\
\code{<token>} "--- токен доступа.\\
\code{<app>} "--- имя приложения, использующего токен.\\
\code{<app\_version>} "--- версия приложения, использующего токен.\\
\code{<os>} "--- операционная система, на которой запущено приложение (желательно с указанием версии).
