\chapter{Обращение к программе}
\label{ch:usage}

Получение данных происходит при помощи HTTP запросов к точкам входа в API\@.

\section{Получение токена}
\label{sec:token}
Чтобы получить доступ к данным студента необходимо сначала получить токен.
Для получения токена используется логин и пароль от ОРИОКС, один студент может получить до восьми токенов.

В запрос входит:\\
\code{<encoded\_auth>} "--- разделённые двоеточием логин:пароль, закодированые в Base64.\\
\code{<app>} "--- имя приложения, использующего токен.\\
\code{<app\_version>} "--- версия приложения, использующего токен.\\
\code{<os>} "--- операционная система, на которой запущено приложение (желательно с указанием версии).

Запрос выглядит следующим образом:
\begin{listing}[H]
  \httpfile[lastline=4]{inc/src/api/token.http}
\end{listing}
\vspace{-0.75cm}

Ответ содержит:\\
\code{token} "--- токен представляет из себя случайную строку длиной в 32 символа (состоит из цифр и латинских букв).

Ответ выглядит следующим образом:
\begin{listing}[H]
  \httpfile[firstline=6]{inc/src/api/token.http}
\end{listing}
\vspace{-0.75cm}

\section{Запрос с использованием токена}
\label{sec:request}
При использовании токена тип аутентификации меняется с Basic на Bearer.
Токен не нужно кодировать в Base64.

\begin{listing}[H]
  \httpfile{inc/src/api/request.http}
\end{listing}
\vspace{-0.75cm}

Все запросы должны содержать токен доступа для идентификации пользователя и подтверждения доступа к ресурсу.

\section{Получение раздела ``Контакты''}
\label{sec:contacts}

Запрос: \code{GET api/v1/contacts}

Формат ответа:
\begin{listing}[H]
  \httpfile[firstline=3,lastline=12]{inc/src/api/contacts.http}
\end{listing}
\vspace{-0.75cm}

Описание полей:\\
\code{classrooom} "--- номер аудитории кафедры.\\
\code{department} "--- кафедра.\\
\code{email} "--- электронная почта кафедры.\\
\code{inner\_tel} "--- внутренний телефон кафедры.\\
\code{tel} "--- телефон кафедры.

Пример ответа:
\begin{listing}[H]
  \httpfile[firstline=14]{inc/src/api/contacts.http}
\end{listing}
\vspace{-0.75cm}

\section{Получение информации о текущей неделе}
\label{sec:faq}

Запрос: \code{GET api/v1/schedule}

Формат ответа:
\begin{listing}[H]
  \httpfile[firstline=3,lastline=7]{inc/src/api/schedule.http}
\end{listing}
\vspace{-0.75cm}

Описание полей:\\
\code{type} "--- тип недели.
Используется для получение расписания группы:
\begin{itemize}
  \item 1: 1 числитель
  \item 2: 1 знаменатель
  \item 3: 2 числитель
  \item 4: 2 знаменатель
\end{itemize}
\code{week} "--- номер недели.

Пример ответа:
\begin{listing}[H]
  \httpfile[firstline=9]{inc/src/api/schedule.http}
\end{listing}
\vspace{-0.75cm}

\section{Получение списка групп}
\label{sec:groups}

Запрос: \code{GET api/v1/schedule/groups}

Формат ответа:
\begin{listing}[H]
  \httpfile[firstline=3,lastline=9]{inc/src/api/groups.http}
\end{listing}
\vspace{-0.75cm}

Описание полей:\\
\code{id} "--- идентификатор группы.\\
\code{name} "--- имя учебной группы.

Пример ответа:
\begin{listing}[H]
  \httpfile[firstline=11]{inc/src/api/groups.http}
\end{listing}
\vspace{-0.75cm}

\section{Получение расписания конкретной группы}
\label{sec:group}

Запрос: \code{GET api/v1/schedule/groups/<id>/types/<type>}

Значение параметров:\\
\code{<id>} "--- идентификатор группы.\\
\code{<type>} "--- тип недели.

Формат ответа:
\begin{listing}[H]
  \httpfile[firstline=3,lastline=17]{inc/src/api/groupSchedule.http}
\end{listing}
\vspace{-0.75cm}

Описание полей:\\
\code{day} "--- номер дня недели.\\
\code{classroom} "--- номер аудитории.\\
\code{name} "--- название предмета.\\
\code{number} "--- номер пары.\\
\code{teacher} "--- ФИО преподавателя.

Пример ответа:
\begin{listing}[H]
  \httpfile[firstline=19]{inc/src/api/groupSchedule.http}
\end{listing}
\vspace{-0.75cm}

\section{Получение информации о студенте}
\label{sec:student}

Запрос: \code{GET api/v1/student}

Формат ответа:
\begin{listing}[H]
  \httpfile[firstline=3,lastline=12]{inc/src/api/student.http}
\end{listing}
\vspace{-0.75cm}

Описание полей:\\
\code{course} "--- курс, на котором сейчас находится студент.\\
\code{full\_name} "--- ФИО студента.\\
\code{group} "--- учебная группа.\\
\code{kaf} "--- кафедра студента.\\
\code{np} "--- направление подготовки.\\
\code{semester} "--- текущий семестр (первый или второй).\\
\code{up} "--- профиль обучения.

Пример ответа:
\begin{listing}[H]
  \httpfile[firstline=14]{inc/src/api/student.http}
\end{listing}
\vspace{-0.75cm}

\section{Получение информации о дисциплинах}
\label{sec:dis}

Запрос: \code{GET api/v1/student/dis?semester=<semester>\&year=<year>}

Значение параметров:\\
\code{<semester>} "--- cеместр, дисциплины за который хотим получить.\\
\code{<year>} "--- год, к которому относится семестр.\\
Если параметры не указаны, будут получены дисциплины за текущий семестр.

Формат ответа:
\begin{listing}[H]
  \httpfile[firstline=3,lastline=16]{inc/src/api/studentDis.http}
\end{listing}
\vspace{-0.75cm}

Описание полей:\\
\code{ball} "--- текущий балл по дисциплине.
\code{dateExam} "--- дата экзамена в формате \code{YYYY-MM-DD} ISO 8601.
\code{formControl} "--- форма контроля.
Возможные значения: Зачёт, Дифференцированный зачёт, Экзамен, Курсовая работа, Курсовой проект, Защита ВКР.\\
\code{id} "--- уникальный идентификатор дисциплины.\\
\code{kaf} "--- кафедра, преподающая предмет.\\
\code{maxBall} "--- максимальный балл по дисциплине.\\
\code{name} "--- название дисциплины.\\
\code{semester} "--- семестр, в котором преподают дисциплину.\\
\code{teachers} "--- ФИО преподавател(я/ей).\\

Пример ответа:
\begin{listing}[H]
  \httpfile[firstline=18]{inc/src/api/studentDis.http}
\end{listing}
\vspace{-0.75cm}
