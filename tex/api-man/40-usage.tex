\chapter{Обращение к программе}
\label{ch:usage}

Получение данных происходит при помощи HTTP запросов к точкам входа в API\@.

\section{Получение токена}
\label{sec:token}
Чтобы получить доступ к данным студента необходимо сначала получить токен.
Для получения токена используется логин и пароль от ОРИОКС, один студент может получить до восьми токенов.

В запрос входит:\\
\code{<encoded\_auth>} "--- разделённые двоеточием логин:пароль, закодированые в Base64.\\
\code{<app>} "--- имя приложения, использующего токен.\\
\code{<app\_version>} "--- версия приложения, использующего токен.\\
\code{<os>} "--- операционная система, на которой запущено приложение (желательно с указанием версии).

Запрос выглядит следующим образом:
\begin{listing}[H]
  \httpfile[lastline=4]{inc/src/api/token.http}
\end{listing}
\vspace{-0.75cm}

Ответ содержит:\\
\code{token} "--- токен представляет из себя случайную строку длиной в 32 символа (состоит из цифр и латинских букв).

Ответ выглядит следующим образом:
\begin{listing}[H]
  \httpfile[firstline=6]{inc/src/api/token.http}
\end{listing}
\vspace{-0.75cm}

\section{Запрос с использованием токена}
\label{sec:request}
При использовании токена тип аутентификации меняется с Basic на Bearer.
Токен не нужно кодировать в Base64.

\begin{listing}[H]
  \httpfile{inc/src/api/request.http}
\end{listing}
\vspace{-0.75cm}

Все запросы должны содержать токен доступа для идентификации пользователя и подтверждения доступа к ресурсу.
