\chapter{Сообщения}
\label{ch:messages}

В ходе работы МП СУПС программисту выдаются сообщения.
Здесь приводится описание их содержания и действия, которые необходимо принять по этим сообщениям.

\section*{Ошибка ``400 Bad Request''}
Возможные описания:
\begin{itemize}
  \item ``Отсутствует обязательный заголовок \{\}'';
  \item ``Недопустимый формат заголовка \{\}'';
  \item ``Отсутствует обязательный ключ \{\}'';
  \item ``Недопустимый формат ключа \{\}''.
\end{itemize}
Где \{\} - заголовок или ключ, вызвавший ошибку.

Для устранения ошибки нужно добавить недостающий заголовок или ключ, или исправить формат уже существующего.

\section*{Ошибка ``401 Unauthorized: Несуществующий или аннулированный токен.''}
Возникает, если передан несуществующий или аннулированный токен в заголовке Authorization.
Для устранения ошибки нужно передать верный токен.

\section*{Ошибка ``403 Forbidden: Один студент может получить не более восьми токенов.''}
Возникает при попытке получить более восьми токенов одним пользователем.
Чтобы устранить ошибку, пользователь должен удалить, которые не используются, освободив этим место для новых токенов.

\section*{Ошибка ``404 Not Found: Отсутствует ресурс по данному URI.''}
Возникает при обращении по несуществующему URL\@.
Для устранения поверьте, что введен правильный URL ресурса.
Если URL верный, следует сообщить об ошибке в поддержку API ОРИОКС\@.

\section*{Ошибка ``405 Method Not Allowed''}
Описание имеет формат: ``Метод \{\} запрещён для данного ресурса''.\\
Где \{\} "--- один из методов: POST, PATCH, PUT, DELETE\@.

Для устранения ошибки следует использовать правильный метод доступа.

\section*{Ошибка ``410 Gone: Данная версия API устарела.''}
Возникает, если происходит обращение к старой версии API, которая более не поддерживается и была удалена.
Для устранения ошибки следует использовать новую версию API\@.
