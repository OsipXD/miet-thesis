\Annotation

В данном документе приведено руководство программиста по использованию ПМ ИСП.
Руководство состоит из пяти разделов.

В разделе ``Назначение и условия применения программы'' указаны назначение и функции, выполняемые программой, условия, необходимые для выполнения программы (объем оперативной памяти, требования к составу и параметрам периферийных устройств, требования к программному обеспечению и т.п.).

В разделе ``Характеристики программы'' приведено описание основных характеристик и особенностей программы (временные характеристики, режим работы, средства контроля правильности выполнения и самовосстанавливаемости программы и т.п.).

В разделе ``Обращение к программе'' приведено описание процедур вызова программы (способы передачи управления и параметров данных и др.).

В разделе ``Входные и выходные данные'' приведено описание организации используемой входной и выходной информации и, при необходимости, ее кодирования.

В разделе ``Сообщения'' указаны тексты сообщений, выдаваемых программисту или оператору в ходе выполнения программы, описание их содержания и действия, которые необходимо предпринять по этим сообщениям.

Оформление программного документа ``Руководство программиста'' произведено по требованиям ЕСПД (ГОСТ 19.101-77, ГОСТ 19.103-77, ГОСТ 19.104-78, ГОСТ 19.105-78, ГОСТ 19.106-78, ГОСТ 19.504-79).
