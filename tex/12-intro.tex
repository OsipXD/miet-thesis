\Introduction
\Abbrev{МП СУПС}{мобильное приложение сопровождения учебного процесса студентов в системе ОРОИКС}
В современных учебных заведениях всё чаще практикуют использование систем сопровождения учебного процесса, которые позволяют учащимся видеть свои оценки, получать домашние задания, узнавать о событиях учебного процесса.
\Abbrev{ОРИОКС}{организация распределенного информационного обмена в корпоративных средах}
В МИЭТе такой системой является ОРИОКС (Организация Распределенного Информационного Обмена в Корпоративных Средах).

Слабым местом ОРИОКС является невозможность оперативного оповещения учащихся о событиях учебного процесса.
\Define{Веб-приложение}{клиент-серверное приложение, в котором клиент взаимодействует с сервером при помощи браузера, а за сервер отвечает — веб-сервер}
Это обусловлено тем, что ОРИОКС представляет из себя веб-приложение и подразумевает регулярное посещение через браузер.
Если же студент долгое время не заходит в систему, что бывает довольно часто, то он может пропустить информацию о важных событиях, например о пересдачах.
Так же, в ОРИОКС невозможна установка напоминаний о предстоящих событиях.

Еще одна проблема веб-приложений – неудобство использования на мобильных
устройствах.
Большая часть студентов посещает ОРИОКС именно с мобильных
устройств, и, хотя, дизайн сайта оптимизирован для работы на небольших экранах, всё же мобильные устройства накладывают свои ограничения.
\Abbrev{ПК}{персональный компьютер}
Скорость интернет соединения, чаще всего, ниже чем на стационарных ПК, кроме того, соединение может часто разрываться, что приводит к дискомфорту
при использовании веб-приложения, т.к.\ для каждой страницы браузер должен
загрузить, помимо данных, разметку и таблицу стилей.


\Define{Мобильное приложение}{программное приложение, предназначенное для работы на смартфонах или планшетных компьютерах}
\Abbrev{API}{application programming interface "--- внешний интерфейс взаимодействия с приложением}
Все эти недостатки можно устранить, создав мобильное приложение, которое будет получать данные от сервера ОРИОКС через API).
Это позволит запрашивать и получать только ту информацию, которая нужна для работы приложения в конкретный момент времени, так как приложению не нужно скачивать таблицу стилей и разметку.
В виду малого потребления трафика приложение сможет работать на смартфоне в фоновом режиме и отображать уведомления и напоминания в тот момент когда они только пришли.

Удобство работы с системой ОРИОКС при нестабильном интернет соединении тоже повысится за счёт меньшего объёма пересылаемых данных.
\Define{Кэширование}{один из способов оптимизации приложений. Заключается в том, чтобы сохранять на некоторое время и переиспользовать результаты медленных операций.}
Кроме того, приложение может кэшировать данные и использовать их даже при отсутствии интернет соединения.

На данный момент нет ни одного приложения работающего с ОРИОКС через API, обладающего полным функционалом и предоставляющего возможности push-уведомлений, поэтому задача является актуальной.
Целью данной работы является создание такого приложения, чтобы повысить оперативность оповещения студентов о событиях учебного процесса.

Задачи выпускной квалификационной работы:
\begin{itemize}
  \item сравнительный анализ существующих программных решений;
  \item выбор платформы;
  \item выбор языка и среды программирования;
  \item разработка алгоритма;
  \item разработка схемы данных;
  \item разработка пользовательского интерфейса;
  \item отладка и тестирование;
  \item разработка руководства оператора.
\end{itemize}

Пояснительная записка состоит из введения, исследовательского, конструкторского и технологического разделов, заключения, списка использованных источников и трёх приложений.

Исследовательский раздел содержит обзор существующих решений, исследование структуры ОРИОКС, описание входных и выходных данных, постановку целей и задач.
Конструкторский раздел содержит обзор и выбор языка программирования, среды разработки, выбор стека технологий, описание алгоритма работы, схему данных и макеты пользовательского интерфейса.
Технологический раздел включает в себя описания применяемых технологий, нюансы разработки под Android, описание процессов сборки, публикации, тестирования, отладки и сопровождения.

В приложении~\ref{ch:appendix1} содержится техническое задание на разработку МП СУПС.
Приложение~\ref{ch:appendix2} содержит фрагменты исходного кода приложения.
В приложении~\ref{ch:appendix3} "--- руководство оператора.
