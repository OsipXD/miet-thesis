\Introduction
\hyphenation{МИЭТе}
В современных учебных заведениях всё чаще практикуют использование систем сопровождения учебного процесса, которые позволяют учащимся видеть свои оценки, домашние задания, узнавать о событиях учебного процесса.
В МИЭТе такой системой является ОРИОКС\@.
\Abbrev{ОРИОКС}{организация распределенного информационного обмена в корпоративных средах}
Рассмотрим существующие недостатки ОРИОКСа.

Слабым местом ОРИОКСа является невозможность оперативного оповещения учащихся о событиях учебного процесса.
Это обусловлено тем, что ОРИОКС представляет из себя веб-приложение
\Define{Веб-приложение}{клиент-серверное приложение, в котором клиент взаимодействует с сервером при помощи браузера, а за сервер отвечает — веб-серверю}
и подразумевает регулярное посещение через браузер.
Если же студент долгое время не заходит в систему, что бывает довольно часто, то он может пропустить информацию о важных событиях, например о пересдачах.
Так же, в ОРИОКС невозможна установка напоминаний о предстоящих событиях.

Еще одна проблема веб-приложений – неудобство использования на мобильных
устройствах.
Большая часть студентов посещает ОРИОКС именно с мобильных
устройств, и, хотя, дизайн сайта оптимизирован для работы на небольших экранах, всё же мобильные устройства накладывают свои ограничения.
Скорость интернет соединения, чаще всего, ниже чем на стационарных ПК,
\Abbrev{ПК}{персональный компьютер}
кроме того, соединение может часто разрываться, что приводит к дискомфорту
при использовании веб-приложения, т.к.\ для каждой страницы браузер должен
загрузить, помимо данных, разметку и таблицу стилей.

Все эти недостатки можно устранить, создав мобильное приложение,
\Define{Мобильное приложение}{программное приложение, предназначенное для работы на смартфонах или планшетных компьютерах}
которое
будет получать данные от сервера ОРИОКС через API).
\Abbrev{API}{application programming interface "--- внешний интерфейс взаимодействия с приложением}
Это позволит запрашивать и получать только ту информацию, которая нужна для работы приложения в конкретный момент времени, так как приложению не нужно скачивать таблицу стилей и разметку.
В виду малого потребления трафика приложение сможет работать на смартфоне в фоновом режиме и отображать уведомления и напоминания в тот момент когда они только пришли.

Удобство работы с системой ОРИОКС при нестабильном интернет соединении тоже повысится за счёт меньшего объёма пересылаемых данных.
Кроме того, приложение может кэшировать
\Define{Кэширование}{один из способов оптимизации приложений. Заключается в том, чтобы сохранять на некоторое время и переиспользовать результаты медленных операций.}
данные и использовать их даже при отсутствии интернет соединения.

На данный момент нет ни одного приложения работающего с ОРИОКС через API\@.
Более того, нет ни одного приложения, позволяющего студентам использовать все возможности ОРИОКС\@.
Целью данной работы является создание такого приложения, чтобы повысить оперативность оповещения студентов о событиях учебного процесса.
